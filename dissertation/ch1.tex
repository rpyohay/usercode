\documentclass[dissertation.tex]{subfiles} 
\begin{document}

\chapter{Introduction}
\label{chap:Introduction}

Although the Standard Model of particle physics has passed every experimental test to date, it leaves some very fundamental questions unanswered.  Why do particles have mass?  Why are their masses so different?  Up to what energy scale is the Standard Model a valid description of nature?  Many competing theories have been proposed to answer these questions.  Establishing the existence of any one of them requires careful searches for deviations from Standard Model predictions of particle production or decay rates.  The stellar performance of the Large Hadron Collider, the 7 TeV center-of-mass energy proton collider located at the European Organization for Nuclear Research (CERN) in Geneva, Switzerland, presents a golden opportunity to do such a search for evidence of new physical phenomena.

One nearly universal prediction of theories of physics beyond the Standard Model is that at a high enough collision energy, heavy particles introduced by the new theory will be produced.  The heavy particles will then decay, leading to distinctive signatures in the hermetic detectors that completely surround the collision points.  By comparing the observed rate of processes with a particular signature to the expected rate from the Standard Model alone, the existence of a particular theory of new physics can be confirmed or ruled out.

This thesis presents a search for evidence of new heavy particles decaying to a final state with two photons, jets, and a striking momentum imbalance that implies the existence of a new kind of particle that can easily pass through matter without leaving a trace.  The signature is motivated by theories that incorporate supersymmetry, a new symmetry of nature that predicts supersymmetric antiparticles to the known particles, just as charge symmetry predicts a positively charged antiparticle for every negatively charged particle and vice versa.  Supersymmetry can provide answers to all of the questions posed in the first paragraph.  Besides its theoretical motivation, the choice of signature is also driven by the low rate of expected Standard Model background.

The search is performed at the Compact Muon Solenoid experiment, a detector capable of identifying photons, electrons, muons, quark jets, $\tau$ leptons, and momentum imbalances with high efficiency.  The central feature of the experiment is a superconducting solenoid, which, at a length of 13 m and a diameter of 7 m \cite{CMS_public}, is one of the largest superconducting magnets ever built.  By bending the paths of charged particles in the final state under the Lorentz force, the magnetic field produced by the solenoid allows charged particle momenta to be accurately measured.  Highly granular calorimeters sit inside the solenoid for the purpose of measuring the energy of neutral final state particles.

This thesis is organized as follows.  Chapters~\ref{chap:Motivation for Physics Beyond the Standard Model} and~\ref{chap:The Supersymmetric Extension to the Standard Model} motivate the search for physics beyond the Standard Model and the specific signature of two photons, as well as give an overview of the Standard Model and supersymmetric theoretical frameworks.  A description is given of the Large Hadron Collider in Chapter~\ref{chap:The Large Hadron Collider} and the Compact Muon Solenoid detector in Chapter~\ref{chap:The Compact Muon Solenoid Experiment}.  Chapters~\ref{chap:Event Selection} and~\ref{chap:Data Analysis} explain in detail the experimental techniques used in the search.  Chapter~\ref{chap:Event Selection} shows how collisions that are likely to have produced a new particle are selected from the enormous amount of data collected, then Chapter~\ref{chap:Data Analysis} shows the data analysis in detail and presents the results.  An interpretation of the results in terms of new physics models is given in Chapter~\ref{chap:Interpretation of Results in Terms of GMSB Models}.  Finally, the thesis is concluded in Chapter~\ref{chap:Conclusion}.

\end{document}