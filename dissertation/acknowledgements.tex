\documentclass[dissertation.tex]{subfiles} 
\begin{document}

\thispagestyle{myheadings}
\markright{\hfill}

\begin{center}
\textbf{Acknowledgements}
\end{center}

This thesis would have been simply impossible without the dedicated efforts of the Large Hadron Collider accelerator team.  I would like to acknowledge the excellent performance of the machine and thank the men and women who worked so hard to build and run it.  Along the same lines, I would also like to thank all the members of the Compact Muon Solenoid collaboration who put in the sweat required to realize the full potential of this amazing detector.  The result arrived at in this thesis is just the last step in a chain of collaborative efforts.

A critical part of my graduate physics education came from working on the installation and commissioning of the electromagnetic calorimeter endcaps.  Special thanks go out to Dave Cockerill, Ken Bell, and Nicolo' Cartiglia for their extraordinary leadership and mentorship.  While working on the LED system software, I frequently sought the advice of Philippe Gras, who was always patient and helpful.

The RA3 analysis group has served simultaneously as a source of new ideas, a sounding board for my own conclusions, and a motivator.  Manfred Paulini, Dave Mason, and Mike Hildreth provided the organizational leadership needed to bring our results to a wider audience, for which I thank them enormously.  I would not have a completed analysis today if it weren't for countless discussions with my fellow RA3 graduate students: Bernadette Heyburn, Robert Stringer, Michael Balazs, Yueh-Feng Liu, and especially Dave Morse and Brian Francis.  Thanks for the insights, the availability, and the lines of code.

The University of Virginia high energy physics group has been an intellectually stimulating home for me these last six years.  First and foremost, I have to thank my advisor Brad Cox for his direction and motivation.  He has given me many golden opportunities to participate in the building of the detector, and encouraged me in an analysis effort that has proved to be extremely academically rewarding.  Of course, I might have blown all of these opportunities without the mentorship and exceptional patience of Sasha Ledovskoy.  I would like to thank Sasha for guiding me through the LED and endcaps work, for introducing me to the analysis software, and for always finding time to answer my questions.  As for the LED work, I must thank Mike Arenton for his technical ability.

One of the most fun and unusual projects I got to work on at UVa was the 4 T magnet.  For his enduring work ethic and attention to detail, I thank Al Tobias.  Without his expertise we were doomed.  My other partner in crime at Maggie's Farm was Dave Phillips, who was also my first HEP lab guru and a good friend.  Thanks also to Chris Neu and Bob Hirosky, who, as well as being expert physicists and wonderful teachers, were great fun to be around.

My life in Charlottesville and at CERN has been enriched indescribably by the friends that I made.  They are no small part of the reason I'm in this field.  Thanks to Sean Lynch, Jamie Antonelli, Jeff Kolb, Doug Berry, Sarah Boutle, Anna Woodard, Tessa Pearson, Jason Slaunwhite, and honorary member Kat Morse of Bldg. 512; Florian Hirsch, Max Scherzer, Heather Gray, Emma Fauss, and Me Linh Jones, my roommates at one time or another; Jirakan Nunkaew and Nick Kvaltine for the fun times in Charlottesville; Justin Griffiths, who might as well be a roommate; Orin Harris for tennis, tuna, and much more; and Toyoko Orimoto, who throws the best parties in Geneva.

None of this would have been possible without the love and support of my family.  Their enthusiasm for my enthusiasm about the Large Hadron Collider has never wavered.  Finally, my deepest thanks go to Ted Kolberg, who has always been on my side.

\clearpage
\end{document}