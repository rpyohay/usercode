\documentclass[dissertation.tex]{subfiles} 
\begin{document}

\chapter{Conclusion}
\label{chap:Conclusion}

The results of a search for evidence of new particle production via final states with 2 photons, large \MET, and $\geq0$ or $\geq1$ jet in $pp$ collisions at $\sqrt{s} = 7$ TeV have been presented.  No deviation in the production rate from that predicted by the Standard Model has been found.  The null results were used to constrain general models of gauge mediated supersymmetry breaking.  In these types of models, gluinos and first- and second-generation squarks are restricted to masses above $\sim1$ TeV.

These bounds on supersymmetry do not exclude it completely.  The gluinos and first- and second-generation squarks can be a little bit heavier (but not too much heavier than a few TeV) and still imply an elegant supersymmetric solution to the hierarchy problem.  More importantly, the bounds on the first- and second-generation squarks say nothing about the stop squark, which is intimately connected to the Higgs mass.  At one loop order in the MSSM, the Higgs mass is given by

%light stop ==> prediction of SUSY and needed to keep Higgs mass light
%125 GeV Higgs ==> stop mass ~760 GeV
%no bounds yet on the stop mass
%future searches could look for light stop production and decay via the NLSP to photons
%or look for Higgsino GMSB sector

\end{document}