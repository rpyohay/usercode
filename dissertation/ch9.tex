\documentclass[dissertation.tex]{subfiles} 
\begin{document}

\chapter{Conclusion}
\label{chap:Conclusion}

The results of a search for evidence of new particle production via final states with 2 photons, large \MET, and $\geq0$ or $\geq1$ jet in $pp$ collisions at $\sqrt{s} = 7$ TeV have been presented.  No deviation in the production rate from that predicted by the Standard Model has been found.  The null results were used to constrain general models of gauge mediated supersymmetry breaking.  In these types of models, gluinos and first- and second-generation squarks are restricted to masses above $\sim1$ TeV.

These bounds on supersymmetry do not exclude it completely.  The gluinos and first- and second-generation squarks can be a little bit heavier (but not too much heavier than a few TeV) and still imply an elegant supersymmetric solution to the hierarchy problem.  More importantly, the bounds on the first- and second-generation squarks say nothing about the stop squark, which is intimately connected to the Higgs mass.  At one loop order in the supersymmetric Standard Model, the lightest Higgs mass is given by \cite{Carena}

\begin{eqnarray}
m_{h}^{2} &\lesssim& m_{Z}^{2} + \frac{3g^{2}m_{t}^{4}}{8\pi^{2}m_{W}^{2}}\left[\ln\frac{M_{S}^{2}}{m_{t}^{2}} + \frac{X_{t}^{2}}{M_{S}^{2}}(1 - \frac{X_{t}^{2}}{12M_{S}^{2}})\right]
\end{eqnarray}
%
where $g$ is the $SU(2)_{L}$ coupling constant, $M_{S}^{2}$ is the average of the two observable stop squared masses, and $X_{t}$ is a parameter that characterizes stop mixing.  The Higgs mass is directly sensitive to the stop mass, for which the only current lower bound of 330 GeV \cite{ATLAS_GMSB_stop_search} is much weaker than for the first- and second-generation squarks (and highly model dependent).  The current hints of a Higgs with mass $\sim125$ GeV \cite{CMS_Higgs, ATLAS_Higgs} point to a stop mass below 2 TeV if SUSY is really a symmetry of nature, depending on model.

Future searches for GMSB could look for either direct pair production of stops decaying via top quarks to neutralinos that then decay to photons, or for stops produced in the decay of a heavier pair-produced particle like the gluino.  Looking for a final state containing a top, antitop, and $\geq1$ photon may be advantageous because the expected SM background should be small.

Top quark reconstruction depends heavily on $b$ jet identification.  The same $b$ tagging techniques needed to find stops could also be applied to a search for a Higgsino-like neutralino decaying primarily to $b\bar{b}$.  If gaugino mixing were in a certain corner of parameter space, then photon + $b\bar{b}$ events might provide a window onto GMSB.

There are a number of interesting possibilities for future GMSB searches in addition to those just outlined.  SUSY searches will likely remain a fruitful avenue of investigation throughout the lifetime of the LHC.

\end{document}