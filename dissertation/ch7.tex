\documentclass[dissertation.tex]{subfiles} 
\begin{document}

\chapter{Data Analysis}
\label{chap:Data Analysis}
%look for high MET excess
%main backgrounds, justify the negligible ones
%explain the control samples
\section{Modeling the QCD Background}
\subsection{Systematic Errors}

\subsubsection{Jet Energy Scale Uncertainty}

The dijet \pT reweighting method utilizes jets corrected for imperfect calorimeter response (see Sec.~\ref{sec:Jets and Missing Transverse Energy} for a description of the jet reconstruction and correction procedure).  Since the applied jet energy scale (JES) factor has an error associated to it due to the limitations of the JES derivation (\cite{CMS_JES_paper} and Sec.~\ref{sec:Jets and Missing Transverse Energy}), this uncertainty must be propagated to the uncertainty on the dijet \pT weights.

The JES contribution to the dijet \pT weights is estimated by performing 1000 pseudo-experiments on each of the $\gamma\gamma$ and ff samples.  For the purpose of estimating the JES error, the results of the true experiment may be thought of as a set of measurements:

\begin{itemize}
  \item The set of \textbf{uncorrected jet 4-vectors} corresponding to the \textbf{leading EM object} in the $\gamma\gamma$ sample $\left\{\mbox{p}_{\mbox{j}1}^{\mu1}, \mbox{p}_{\mbox{j}1}^{\mu2},...,\mbox{p}_{\mbox{j}1}^{\mu \mbox{N}_{\gamma\gamma}}\right\}$
  \item The set of \textbf{uncorrected jet 4-vectors} corresponding to the \textbf{trailing EM object} in the $\gamma\gamma$ sample $\left\{\mbox{p}_{\mbox{j}2}^{\mu1}, \mbox{p}_{\mbox{j}2}^{\mu2},...,\mbox{p}_{\mbox{j}2}^{\mu \mbox{N}_{\gamma\gamma}}\right\}$
  \item The set of \textbf{JES} accompanying the uncorrected jet 4-vectors corresponding to the \textbf{leading EM object} in the $\gamma\gamma$ sample $\left\{\mbox{c}_{\mbox{j}1}^{1}, \mbox{c}_{\mbox{j}1}^{2},...,\mbox{c}_{\mbox{j}1}^{\mbox{N}_{\gamma\gamma}}\right\}$
  \item The set of \textbf{JES} accompanying the uncorrected jet 4-vectors corresponding to the \textbf{trailing EM object} in the $\gamma\gamma$ sample $\left\{\mbox{c}_{\mbox{j}2}^{1}, \mbox{c}_{\mbox{j}2}^{2},...,\mbox{c}_{\mbox{j}2}^{\mbox{N}_{\gamma\gamma}}\right\}$
  \item The set of \textbf{JES uncertainties} accompanying the uncorrected jet 4-vectors corresponding to the \textbf{leading EM object} in the $\gamma\gamma$ sample $\left\{\sigma_{\mbox{cj}1}^{1}, \sigma_{\mbox{cj}1}^{2},...,\sigma_{\mbox{cj}1}^{\mbox{N}_{\gamma\gamma}}\right\}$
  \item The set of \textbf{JES uncertainties} accompanying the uncorrected jet 4-vectors corresponding to the \textbf{trailing EM object} in the $\gamma\gamma$ sample $\left\{\sigma_{\mbox{cj}2}^{1}, \sigma_{\mbox{cj}2}^{2},...,\sigma_{\mbox{cj}2}^{\mbox{N}_{\gamma\gamma}}\right\}$
  \item The set of \textbf{uncorrected jet 4-vectors} corresponding to the \textbf{leading EM object} in the ff sample $\left\{\mbox{p}_{\mbox{j}1}^{\mu1}, \mbox{p}_{\mbox{j}1}^{\mu2},...,\mbox{p}_{\mbox{j}1}^{\mu \mbox{N}_{\mbox{ff}}}\right\}$
  \item The set of \textbf{uncorrected jet 4-vectors} corresponding to the \textbf{trailing EM object} in the ff sample $\left\{\mbox{p}_{\mbox{j}2}^{\mu1}, \mbox{p}_{\mbox{j}2}^{\mu2},...,\mbox{p}_{\mbox{j}2}^{\mu \mbox{N}_{\mbox{ff}}}\right\}$
  \item The set of \textbf{JES} accompanying the uncorrected jet 4-vectors corresponding to the \textbf{leading EM object} in the ff sample $\left\{\mbox{c}_{\mbox{j}1}^{1}, \mbox{c}_{\mbox{j}1}^{2},...,\mbox{c}_{\mbox{j}1}^{\mbox{N}_{\mbox{ff}}}\right\}$
  \item The set of \textbf{JES} accompanying the uncorrected jet 4-vectors corresponding to the \textbf{trailing EM object} in the ff sample $\left\{\mbox{c}_{\mbox{j}2}^{1}, \mbox{c}_{\mbox{j}2}^{2},...,\mbox{c}_{\mbox{j}2}^{\mbox{N}_{\mbox{ff}}}\right\}$
  \item The set of \textbf{JES uncertainties} accompanying the uncorrected jet 4-vectors corresponding to the \textbf{leading EM object} in the ff sample $\left\{\sigma_{\mbox{cj}1}^{1}, \sigma_{\mbox{cj}1}^{2},...,\sigma_{\mbox{cj}1}^{\mbox{N}_{\mbox{ff}}}\right\}$
  \item The set of \textbf{JES uncertainties} accompanying the uncorrected jet 4-vectors corresponding to the \textbf{trailing EM object} in the ff sample $\left\{\sigma_{\mbox{cj}2}^{1}, \sigma_{\mbox{cj}2}^{2},...,\sigma_{\mbox{cj}2}^{\mbox{N}_{\mbox{ff}}}\right\}$
\end{itemize}
%
From these measurements, the $\gamma\gamma$ and ff dijet \pT spectra and the resulting ff dijet weights can be calculated.  In each of the 1000 pseudo-experiments, a new set of JES factors is generated according to the measured JES uncertainties, and new dijet \pT spectra and weights are subsequently calculated.  The spread of the 1000 weights (binned in dijet \pT) is taken as the error due to JES uncertainty.  The total error on the weights is the quadrature sum of the JES error and the statistical error, and is propagated to the error on the final \MET measurement via a similar pseudo-experiment procedure described in Sec.~\ref{sec:Statistical Uncertainty in the ff or ee Weights}.\footnote{The \MET is uncorrected and therefore its central value per event is unaffected by a change in the JES.}

%do this check
If the JES uncertainty were to cause the jet energy to be reconstructed below the 20 GeV ntuple cut, there could be a small error or bias in the \MET introduced due to EM-matched jets falling below the matching threshold.  The percentage of jets lost due to jet \ET matching threshold has been checked in data, and found to be X\% (X\% of events).  Furthermore, the trailing EM \ET cut is 25 GeV/c, implying that the JES would have to be mis-measured by at least 20\% to fall below the jet matching threshold.  Since the typical JES uncertainty is no more than 5\%, a mis-measurement of this type is a 4$\sigma$ event and should occur in only 0.1\% of cases.  As expected, this effect is negligible, as shown in Figure X.

%\begin{figure}
%	\centering
%	\subfloat[ff dijet \pT weights including effect of jets falling below the matching threshold.]{\label{fig:dummy}\includegraphics[scale=0.7]{dummy}}
%	\hspace{1cm}
%	\subfloat[ff dijet \pT weights neglecting effect of jets falling below the matching threshold.]{\label{fig:dummy}\includegraphics[scale=0.7]{dummy}}
%	\hspace{1cm}
%	\subfloat[Percentage difference between JES errors shown in (a) and (b).]{\label{fig:dummy}\includegraphics[scale=0.7]{dummy}}
%	\caption{ff dijet \pT weights with JES error.}
%\end{figure}

\subsubsection{Statistical Uncertainty in the ff or ee Weights}
\label{sec:Statistical Uncertainty in the ff or ee Weights}

\section{Modeling the Electroweak Background}
\section{Results}
%including all uncertainties
%reminder table of all uncertainties that are described more thoroughly in previous sections

Lorum ipsum fuck Republicans.

\end{document}