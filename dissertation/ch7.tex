\documentclass[dissertation.tex]{subfiles} 
\begin{document}

\chapter{Data Analysis}
\label{chap:Data Analysis}

%update the int. lumi. everywhere
\textcolor{red}{\textbf{Update the int. lumi. everywhere.}}

The signature of GGM SUSY particle production in this search is an excess of two-photon events with high \MET.  \MET is reconstructed using the particle flow algorithm as described in Sec.~\ref{sec:MET}.  Candidate two-photon events, as well as control events, are selected according to the offline object criteria presented in Secs.~\ref{sec:Photons} and~\ref{sec:Electrons}, the event quality criteria in Sec.~\ref{sec:Event Quality}, and the trigger requirements in Sec.~\ref{sec:HLT}.  These are summarized in Table~\ref{tab:selection_summary}.

\begin{table}[hcbp]
\caption{Selection criteria for $\gamma\gamma$, $e\gamma$, $ee$, and $\mathit{ff}$ events.}
\centering
\begin{tabular}{|c|c|c|c|c|}
\hline
\multirow{2}{*}{Variable} & \multicolumn{4}{c|}{Cut} \\
\cline{2-5}
& $\gamma\gamma$ & $e\gamma$ & $ee$ & $\mathit{ff}$ \\
\hline
\hline
%HLT & \begin{tabular}[c]{@{}c@{}}26IsoVL/\\18\\\\36CaloIdL/\\22CaloIdL\\\\36CaloIdLIsoVL/\\22CaloIdLIsoVL\end{tabular} & \begin{tabular}[c]{@{}c@{}}26IsoVL/\\18\\\\36CaloIdL/\\22CaloIdL\\\\36CaloIdLIsoVL/\\22CaloIdLIsoVL\end{tabular} & \begin{tabular}[c]{@{}c@{}}26IsoVL/\\18\\\\36CaloIdL/\\22CaloIdL\\\\36CaloIdLIsoVL/\\22CaloIdLIsoVL\end{tabular} & \begin{tabular}[c]{@{}c@{}}26IsoVL/\\18\\\\36CaloIdL/\\22CaloIdL\\\\36CaloIdLIsoVL/\\22CaloIdLIsoVL\\\\36CaloIdLIsoVL/\\22R9Id\\\\36R9Id/\\22CaloIdLIsoVL\\\\36R9Id/\\22R9Id\end{tabular} \\
%\hline
HLT match & IsoVL & IsoVL & IsoVL & IsoVL $||$ R9Id \\
\hline
$E_{T}$ & \begin{tabular}[c]{@{}c@{}}$> 40$/\\$> 25$ GeV\end{tabular} & \begin{tabular}[c]{@{}c@{}}$> 40$/\\$> 25$ GeV\end{tabular} & \begin{tabular}[c]{@{}c@{}}$> 40$/\\$> 25$ GeV\end{tabular} & \begin{tabular}[c]{@{}c@{}}$> 40$/\\$> 25$ GeV\end{tabular} \\
\hline
SC $|\eta|$ & $< 1.4442$ & $<1.4442$ & $< 1.4442$ & $<1.4442$ \\
\hline
$H/E$ & $<0.05$ & $<0.05$ & $<0.05$ & $<0.05$ \\
\hline
$R9$ & $< 1$ & $< 1$ & $< 1$ & $< 1$ \\
\hline
Pixel seed & No/No & Yes/No & Yes/Yes & No/No \\
\hline
$I_{\mathrm{comb}}$, $\sigma_{i\eta i\eta}$ & \begin{tabular}[c]{@{}c@{}}$< 6$ GeV \&\&\\$< 0.011$\end{tabular} & \begin{tabular}[c]{@{}c@{}}$< 6$ GeV \&\&\\$< 0.011$\end{tabular} & \begin{tabular}[c]{@{}c@{}}$< 6$ GeV \&\&\\$< 0.011$\end{tabular} & \begin{tabular}[c]{@{}c@{}}$< 20$ GeV \&\&\\($\geq 6$ GeV $||$\\$\geq 0.011$)\end{tabular} \\
\hline
JSON & Yes & Yes & Yes & Yes \\
\hline
No. good PVs & $\geq 1$ & $\geq 1$ & $\geq 1$ & $\geq 1$ \\
\hline
$\Delta R_{\mathrm{EM}}$ & $> 0.6$ & $> 0.6$ & $> 0.6$ & $> 0.6$ \\
\hline
$\Delta\phi_{\mathrm{EM}}$ & $\geq 0.05$ & $\geq 0.05$ & $\geq 0.05$ & $\geq 0.05$ \\
\hline
\end{tabular}
\label{tab:selection_summary}
\end{table}

This search utilizes 4.7 $\mbox{fb}^{-1}$ of CMS data collected during the period April-December 2011, corresponding to the following datasets \cite{DAS}:

\begin{itemize}
\item \verb+/Photon/Run2011A-05Jul2011ReReco-ECAL-v1/AOD+
\item \verb+/Photon/Run2011A-05Aug2011-v1/AOD+
\item \verb+/Photon/Run2011A-03Oct2011-v1/AOD+
\item \verb+/Photon/Run2011B-PromptReco-v1/AOD+
\end{itemize}

The search strategy is to model the backgrounds to the GGM SUSY signal using \MET shape templates derived from the control samples, and then to look for a high-\MET excess above the estimated background in the $\gamma\gamma$ sample.  There are two categories of backgrounds: QCD processes with no real \MET and electroweak processes with real \MET from neutrinos.  The relevant QCD background processes are multijet production with at least two jets faking photons, photon + jet production with at least one jet faking a photon, diphoton production, and $Z$ production with a radiated photon where at least one of the $Z$ decay products (typically a jet) fakes a photon.  The relevant electroweak background processes, which are small compared to the QCD background, involve $W\rightarrow e\nu$ decay with a recoiling jet that fakes a photon or a real radiated photon (the $W$ may come from the decay of a top quark in $t\bar{t}$ events).  In both cases, the electron is misidentified as a photon due to a small inefficiency in reconstructing the electron pixel seed.  The main diagrams contributing to the QCD(electroweak) backgrounds are shown in Figure~\ref{fig:QCD_background_diagrams}(\ref{fig:EW_background_diagrams}).  \textcolor{red}{\textbf{Generate these Feynman diagrams.}}

%QCD EM-enriched (isub=11,12,13,28,53,68) (mstp82=4 ==> multiple int. assuming varying impact parameter and hadronic overlap consistent with a double gaussian matter distribution, but turning off at low pt)
%q_i q_j --> q_i q_j
%q_i q_i --> q_j q_j
%q_i q_i_bar --> g g
%q_i g --> q_i g
%g g --> q_i q_i_bar
%g g --> g g
%photon + jet EM enriched (isub=14,18,29)
%q_i q_i_bar --> g gamma
%q_i q_i_bar --> gamma gamma
%q_i g --> q_i gamma
%diphoton + jets (pp-->gammagamma, +1 j, +2 j (j ==> sum over gluons and light quarks, not t/b)), tree level to order X (QCD) and Y (QED)
%DiPhotonBorn: q_i q_i_bar --> gamma gamma
%DiPhotonBox:  g g --> gamma gamma
%W + jets
%W + photon
%Z + photon
%

%Wgg, Zgg (negligible)

%\begin{figure}
%	\centering
%	\subfloat[Dijet production via $gg$ and $qg$ interactions.]{\label{fig:dijet}\includegraphics[scale=1.35]{dijet}}
%	\hspace{1cm}
%	\subfloat[Photon + jet production via $qg$ interactions.]{\label{fig:photon_jet}\includegraphics[scale=1.35]{photon_jet}}
%	\\
%	\subfloat[Diphoton production via $q\bar{q}$ and $gg$ interactions.]{\label{fig:diphoton}\includegraphics[scale=1.8]{diphoton}}
%	\hspace{1cm}
%	\subfloat[$Z\gamma$ production.]{\label{fig:Zgamma}\includegraphics[scale=1.8]{Zgamma}}
%	\caption{Representative Feynman diagrams of some QCD backgrounds to the GGM SUSY search with instrumental effects described.}
%	\label{fig:QCD_background_diagrams}
%\end{figure}
%
%\begin{figure}
%	\centering
%	\subfloat[$W\gamma$ production.]{\label{fig:Wgamma}\includegraphics[scale=1.35]{Wgamma}}
%	\hspace{1cm}
%	\subfloat[$W$ + jet production.]{\label{fig:W_jet}\includegraphics[scale=1.35]{W_jet}}
%	\hspace{1cm}
%	\subfloat[$t\bar{t}$ production.]{\label{fig:ttbar}\includegraphics[scale=1.35]{ttbar}}
%	\caption{Representative Feynman diagrams of some electoweak backgrounds to the GGM SUSY search with instrumental effects described.}
%	\label{fig:EW_background_diagrams}
%\end{figure}

Figure~\ref{fig:MET_data_vs_MC_backgrounds} shows the \MET spectrum of the $\gamma\gamma$ search data sample overlaid on the \MET spectra of MC simulated background components.  The MC spectra are normalized to the integrated luminosity of the $\gamma\gamma$ data sample.  \textcolor{red}{\textbf{Make this plot.}}  The dominant background components are QCD inclusive photon processes.  The MC is not used in the actual background estimation.  It is just shown here to illustrate the breakdown of backgrounds.

%\begin{figure}
%	\centering
%	\includegraphics[scale=1.35]{MET_data_vs_MC_backgrounds}
%	\caption{\MET spectrum of the $\gamma\gamma$ search data sample overlaid on the \MET spectra of MC simulated background components.  The MC spectra are normalized to the integrated luminosity of the $\gamma\gamma$ data sample.  A description of the MC samples used may be found in Appendix~\ref{chap:Monte Carlo Samples}}.
%	\label{fig:MET_data_vs_MC_backgrounds}
%\end{figure}

Data control samples are used to model all of the backgrounds.  The primary control sample used to model the QCD background is the $\mathit{ff}$ sample, which is similar to the candidate $\gamma\gamma$ sample but with combined isolation or $\sigma_{i\eta i\eta}$ cuts inverted.  The cuts on these variables are used to distinguish between photons and jets, so by inverting those cuts, the resulting $\mathit{ff}$ sample becomes enriched with QCD dijets.  Because the fake photons are still required to pass a tight cut on $H/E$, they are guaranteed to be very electromagnetic jets, with an EM energy scale and resolution similar to that of the candidate photons.  This insures that the resulting estimate of the \MET shape does not have too long of a tail from severe HCAL mis-measurements that are actually rare in the $\gamma\gamma$ sample, as shown in Figure~\ref{fig:MET_MC_ggVsFF_varyHOverE}.  \textcolor{red}{\textbf{Plot the $\gamma\gamma$/$\mathit{ff}$ \MET agreement for different values of the ff $H/E$ cut in MC.  Make the same plot in data for a restricted \MET range?}}

%\begin{figure}
%	\centering
%	\subfloat[MC.  See App.~\ref{chap:Monte Carlo Samples} for the MC samples used.]{\label{fig:MET_MC_ggVsFF_varyHOverE_MC}\includegraphics[scale=1.35]{MET_MC_ggVsFF_varyHOverE_MC}}
%	\hspace{1cm}
%	\subfloat[Data, restricted to $\MET < 100$ GeV.]{\label{fig:MET_MC_ggVsFF_varyHOverE_data}\includegraphics[scale=1.35]{MET_MC_ggVsFF_varyHOverE_data}}
%	\caption{\MET spectra of the $\gamma\gamma$ and $\mathit{ff}$ samples for various values of the $\mathit{ff} H/E$ cut.}
%	\label{fig:MET_MC_ggVsFF_varyHOverE}
%\end{figure}

As a cross-check, the $ee$ sample is also used to model the QCD background.  This sample of $Z$ decays should have no true \MET, just like the $\mathit{ff}$ sample, and the electron definition (differing from the photon definition only in the presence of a pixel seed) insures that the electron energy scale and resolution is similar to that of the photon.

Finally, the $e\gamma$ sample is used to model the electroweak background from $W\rightarrow e\nu$ decays.  The $e\gamma$ \MET distribution is scaled by the electron$\rightarrow$photon misidentification rate to predict the number of $W\gamma$, $W$ + jet, and $t\bar{t}$ events in the $\gamma\gamma$ sample.

The remainder of this chapter describes the data analysis procedures and the final results of the search.  Sec.~\ref{sec:Modeling the QCD Background} addresses the QCD background estimation.  Sec.~\ref{sec:Modeling the Electroweak background} addresses the electroweak background estimation.  The chapter concludes with a discussion of systematic errors in Sec.~\ref{sec:Systematic Errors} and a presentation of the final results in Sec.~\ref{sec:Results}.

\section{Modeling the QCD Background}
\label{sec:Modeling the QCD Background}

\subsection{Outline of the Procedure}
\label{sec:Outline of the Procedure}

Due to the fact that the CMS ECAL energy resolution is much better than the HCAL energy resolution, the energies of the two candidate photons in the events of the $\gamma\gamma$ sample are typically measured to far greater accuracy and precision than the energy of the hadronic recoil in those events.  Therefore, fake \MET in the $\gamma\gamma$ sample is almost entirely the result of hadronic mis-measurement in QCD dijet, photon + jet, and diphoton events.  The strategy employed to model this background is to find a control sample in data consisting of two well-measured EM objects, just like the candidate $\gamma\gamma$ sample, and assign each event a weight to account for the underlying kinematic differences between the control and candidate samples.  Once the reweighted \MET spectrum of the control sample is created, it is then normalized in the low-\MET region, the assumption being that GGM SUSY does not predict a significant amount of events at low \MET.  There are three aspects to this QCD background estimation procedure that bear highlighting:

\begin{description}
\item[Choice of control sample] Since the underlying cause of \MET in the candidate sample is mis-measured hadronic activity, a control sample with similar hadronic activity to the candidate sample should be chosen.  Hadronic activity refers to number of jets, jet $E_{T}$, pileup, etc.
\item[Reweighting] The control sample is reweighted so that its \MET spectrum appears as it would if the control sample had the same kinematic properties as the candidate sample (i.e. particle $p_{T}$ and $\eta$ distributions, etc.).  By choosing an appropriate control sample and reweighting it, the control \MET distribution should now match both the hadronic activity and the kinematics of the candidate sample.
\item[Normalization] Finally, the control \MET distribution is normalized in a region of low \MET, where contamination from the expected GGM SUSY signal is small.  This implies an extrapolation of the low-\MET QCD background prediction to the high-\MET signal region.
\end{description}

As explained in the beginning of this chapter, the $\mathit{ff}$ sample is used as the primary QCD control sample, while the $ee$ sample is used as a cross-check.  Both samples have two well-measured EM objects per event, no real \MET, and similar hadronic activity to the $\gamma\gamma$ sample.  Figure~\ref{fig:hadronic_activity} shows a comparison of the shapes of some distributions relevant to hadronic activity between the $\gamma\gamma$, $ee$, and $\mathit{ff}$ samples.  In general, the $ee$ sample has less hadronic activity than the $\gamma\gamma$ and $\mathit{ff}$ samples, as shown by the more steeply falling $ee$ distributions in Figs.~\ref{fig:hadronic_activity_HT},~\ref{fig:hadronic_activity_Nj},~\ref{fig:hadronic_activity_MHT}, and~\ref{fig:hadronic_activity_j1ET}.  In addition to the kinematic reweighting, there is also a reweighting by number of jets per event, which attempts to correct for this difference (see Sec.~\ref{sec:Reweighting}).

%include similar figures for MC?
\begin{figure}
	\centering
	\subfloat[$H_{T}$, defined as the scalar sum of corrected jet $E_{T}$ for jets defined as in Table~\ref{tab:jet_definition}.]{\label{fig:hadronic_activity_HT}\includegraphics[scale=0.25]{hadronic_activity_HT}}
	\hspace{0.75cm}
	\subfloat[Number of jets per event for jets defined as in Table~\ref{tab:jet_definition}.]{\label{fig:hadronic_activity_Nj}\includegraphics[scale=0.25]{hadronic_activity_Nj}}
	\hspace{0.75cm}
	\subfloat[$\not\!\! H_{T}$, defined as the magnitude of the negative vectorial sum of corrected jet $E_{T}$ for jets defined as in Table~\ref{tab:jet_definition}.]{\label{fig:hadronic_activity_MHT}\includegraphics[scale=0.25]{hadronic_activity_MHT}}
	\\
	\subfloat[Corrected $E_{T}$ for the jet with the largest corrected $E_{T}$ per event, for jets defined as in Table~\ref{tab:jet_definition} (excluding the $p_{T}$ requirement).]{\label{fig:hadronic_activity_j1ET}\includegraphics[scale=0.25]{hadronic_activity_j1ET}}
	\hspace{0.75cm}
	\subfloat[$\rho$ (average pileup energy density in the calorimeters per unit $\eta\cdot\phi$, cf. Sec.~\ref{sec:Photons}).]{\label{fig:hadronic_activity_rho}\includegraphics[scale=0.25]{hadronic_activity_rho}}
	\hspace{0.75cm}
	\subfloat[Number of good reconstructed primary vertices per event according to the criteria of Sec.~\ref{sec:Event Quality}.]{\label{fig:hadronic_activity_nPV}\includegraphics[scale=0.25]{hadronic_activity_nPV}}
	\caption{Comparison of the shapes of some distributions relevant to hadronic activity between the $\gamma\gamma$, $ee$ (81 GeV $\leq m_{\mathrm{ee}} <$ 101 GeV), and $\mathit{ff}$ samples.  The $ee$ and $\mathit{ff}$ distributions are normalized to the number of events in the $\gamma\gamma$ distribution.  Errors are statistical only.}
	\label{fig:hadronic_activity}
\end{figure}

%add a footnote describing the PF electron and PF muon definitions, with references
\begin{table}[hcbp]
\caption{Definition of HB/HE/HF hadronic jets.  \textcolor{red}{\textbf{Add a footnote describing the PF electron and PF muon definitions, with references.}}}
\centering
\begin{tabular}{|c|c|}
\hline
Variable & Cut \\
\hline
\hline
Algorithm & \verb+L1FastL2L3Residual+ corrected PF (cf. Sec.~\ref{sec:Jets and Missing Transverse Energy}) \\
\hline
$p_{T}$ & $> 30$ GeV \\
\hline
$|\eta|$ & $< 5.0$ \\
\hline
\begin{tabular}[c]{@{}c@{}}Neutral hadronic\\energy fraction\end{tabular} & $< 0.99$ \\
\hline
\begin{tabular}[c]{@{}c@{}}Neutral electromagnetic\\energy fraction\end{tabular} & $< 0.99$ \\
\hline
Number of constituents & $> 1$ \\
\hline
Charged hadronic energy & $> 0.0$ GeV if $|\eta|  < 2.4$ \\
\hline
Number of charged hadrons & $> 0$ if $|\eta|  < 2.4$ \\
\hline
\begin{tabular}[c]{@{}c@{}}Charged electromagnetic\\energy fraction\end{tabular} & $< 0.99$ if $|\eta| < 2.4$ \\
\hline
\begin{tabular}[c]{@{}c@{}}$\Delta R$ to nearest electron, muon,\\or one of the two\\primary EM objects\end{tabular} & $> 0.5$ \\
\hline
\end{tabular}
\label{tab:jet_definition}
\end{table}

\subsection{Reweighting}
\label{sec:Reweighting}

To reweight the control sample events to match the kinematics of the candidate sample events, a weight based on the $p_{T}$ of the di-EM-object system and the number of jets in the event is used.  As explained in Sec.~\ref{sec:Outline of the Procedure}, \MET in the $\gamma\gamma$, $\mathit{ff}$, and $ee$ samples is due to the poorly measured hadronic recoil off the well-measured di-EM system.  Therefore, the $p_{T}$ of the di-EM system is a good handle on the true magnitude of the hadronic recoil, which affects the measured \MET.  The di-EM system is depicted in Figure~\ref{fig:di-EM_pT_cartoon}.

\begin{figure}
	\centering
	\includegraphics[scale=0.5]{di-EM_pT_cartoon}
	\caption{Cartoon showing the di-EM system in blue and the hadronic recoil in black.  The di-EM $p_{T}$ (dashed blue line) is used to reweight the control sample kinematic properties to match those of the candidate $\gamma\gamma$ sample.}.
	\label{fig:di-EM_pT_cartoon}
\end{figure}

Whereas the di-EM $p_{T}$ reweighting accounts for differences in production kinematics between the control and $\gamma\gamma$ samples, a simultaneous reweighting based on the number of jets in the event accounts for differences in hadronic activity between the samples, especially between $ee$ and $\gamma\gamma$ (cf. Fig.~\ref{fig:hadronic_activity}).  Jets are defined as in Table~\ref{tab:jet_definition_for_Nj_reweighting}.  Figure~\ref{fig:dijet_pT_and_Nj_vs_dijet_pT_reweighting} shows the effect of reweighting by number of jets per event, which is to increase(decrease) the tail of the $ee$($\mathit{ff}$) \MET spectrum.

%include ee/gg and ff/gg agreement?
%include similar figures for MC?
\begin{figure}
	\centering
	\subfloat[$ee$ \MET spectra.]{\label{fig:ee_dijet_pT_and_Nj_vs_dijet_pT_reweighting}\includegraphics[scale=0.3]{ee_dijet_pT_and_Nj_vs_dijet_pT_reweighting}}
	\hspace{1cm}
	\subfloat[$\mathit{ff}$ \MET spectra.]{\label{fig:ff_dijet_pT_and_Nj_vs_dijet_pT_reweighting}\includegraphics[scale=0.3]{ff_dijet_pT_and_Nj_vs_dijet_pT_reweighting}}
	\\
	\subfloat[Ratio of $ee$ to $\mathit{ff}$ \MET spectra.]{\label{fig:ee_over_ff_dijet_pT_and_Nj_vs_dijet_pT_reweighting}\includegraphics[scale=0.3]{ee_over_ff_dijet_pT_and_Nj_vs_dijet_pT_reweighting}}
	\hspace{1cm}
	\subfloat[Ratio of $ee$ to $\mathit{ff}$ \MET spectra, zoomed x-axis.]{\label{fig:ee_over_ff_dijet_pT_and_Nj_vs_dijet_pT_reweighting_zoom}\includegraphics[scale=0.3]{ee_over_ff_dijet_pT_and_Nj_vs_dijet_pT_reweighting_zoom}}
	\caption{\MET spectra of the reweighted $ee$ (81 GeV $\leq m_{\mathrm{ee}} <$ 101 GeV) and $\mathit{ff}$ control samples.  Blue squares indicate di-EM $p_{T}$ reweighting only; red triangles indicate di-EM $p_{T}$ + number of jets reweighting.  PF $p_{T}$ (cf. p.~\pageref{fig:ET_bias_vs_EMF}) is used to calculate the di-EM $p_{T}$.  The full normalization procedure is employed, along with $ee$ sideband subtraction (discussed in at the end of this section).  Error bars include statistical, reweighting, and normalization error (see Sec.~\ref{sec:Systematic Errors}).}
	\label{fig:dijet_pT_and_Nj_vs_dijet_pT_reweighting}
\end{figure}

Although the electron and photon energies are well measured by the ECAL, the ECAL-only measurement of the fake photon energy (cf. Sec~\ref{sec:Photons}) is biased slightly low due to the fact that fakes (which are really jets) tend to deposit some energy in the HCAL.  This can be seen in Figs.~\ref{fig:ET_bias_vs_EMF} and~\ref{fig:4684pb-1_single_ETBias}, which show the relative difference between the ECAL-only $E_{T}$ measurement and the PF $E_{T}$ measurement vs. EMF for electrons, photons, and fakes.  PF $E_{T}$ is defined as the \verb+L1Fast+-corrected $E_{T}$ of the nearest PF jet with $p_{T} \geq 20$ GeV (i.e., the $E_{T}$ of the PF jet object reconstructed from the same ECAL shower as the fake photon).  On average, the fakes tend to deposit a few percent more energy in the HCAL than the electrons or photons, which is recovered by the PF algorithm.  For this reason, the PF $p_{T}$ is used in the calculation of di-EM $p_{T}$ rather than the ECAL-only $p_{T}$.\footnote{In the few events ($\mathcal{O}(10^{-3})$) in which two PF jet objects, corresponding to the two electrons or fakes, are not found, the ECAL-only $p_{T}$ is used.}  This leads to a modest improvement in the agreement between the $ee$ and $\mathit{ff}$ \MET spectra, as shown in Figure~\ref{fig:ee_vs_ff_di-EM_vs_dijet_pT_reweighting}.

%include similar figures for MC?
\begin{figure}
	\centering
	\subfloat[Leading electron in $ee$ events.]{\label{fig:4684pb-1_eLeading_ETBias_log}\includegraphics[scale=0.25]{4684pb-1_eLeading_ETBias_log}}
	\hspace{1cm}
	\subfloat[Trailing electron in $ee$ events.]{\label{fig:4684pb-1_eTrailing_ETBias_log}\includegraphics[scale=0.25]{4684pb-1_eTrailing_ETBias_log}}
	\\
	\subfloat[Leading fake in $\mathit{ff}$ events.]{\label{fig:4684pb-1_fLeading_ETBias_log}\includegraphics[scale=0.25]{4684pb-1_fLeading_ETBias_log}}
	\hspace{1cm}
	\subfloat[Trailing fake in $\mathit{ff}$ events.]{\label{fig:4684pb-1_fTrailing_ETBias_log}\includegraphics[scale=0.25]{4684pb-1_fTrailing_ETBias_log}}
	\\
	\subfloat[Leading photon in $\gamma\gamma$ events.]{\label{fig:4684pb-1_gLeading_ETBias_log}\includegraphics[scale=0.25]{4684pb-1_gLeading_ETBias_log}}
	\hspace{1cm}
	\subfloat[Trailing photon in $\gamma\gamma$ events.]{\label{fig:4684pb-1_gTrailing_ETBias_log}\includegraphics[scale=0.25]{4684pb-1_gTrailing_ETBias_log}}
	\caption{Relative difference between the ECAL-only $E_{T}$ measurement and the PF $E_{T}$ measurement vs. EMF.  PF $E_{T}$ is defined in the text.}
	\label{fig:ET_bias_vs_EMF}
\end{figure}

%include similar figures for MC?
\begin{figure}
	\centering
	\includegraphics[scale=0.5]{4684pb-1_single_ETBias}
	\caption{Average relative difference between the ECAL-only $E_{T}$ measurement and the PF $E_{T}$ measurement vs. EMF for the leading (filled marker) and trailing (open marker) electrons in $ee$ events (red triangles), fakes in $\mathit{ff}$ events (blue squares), and photons in $\gamma\gamma$ events (black circles).  These are nothing more than profile histograms of Fig.~\ref{fig:ET_bias_vs_EMF}.  PF $E_{T}$ is defined in the text.  Error bars are statistical only.}
	\label{fig:4684pb-1_single_ETBias}
\end{figure}

%include ee/gg and ff/gg agreement?
%include similar figures for MC?
\begin{figure}
	\centering
	\subfloat[$ee$ \MET spectra.]{\label{fig:ee_di-EM_vs_dijet_pT_reweighting}\includegraphics[scale=0.3]{ee_di-EM_vs_dijet_pT_reweighting}}
	\hspace{1cm}
	\subfloat[$\mathit{ff}$ \MET spectra.]{\label{fig:ff_di-EM_vs_dijet_pT_reweighting}\includegraphics[scale=0.3]{ff_di-EM_vs_dijet_pT_reweighting}}
	\\
	\subfloat[Ratio of $ee$ to $\mathit{ff}$ \MET spectra.]{\label{fig:ee_over_ff_di-EM_vs_dijet_pT_reweighting}\includegraphics[scale=0.3]{ee_over_ff_di-EM_vs_dijet_pT_reweighting}}
	\hspace{1cm}
	\subfloat[Ratio of $ee$ to $\mathit{ff}$ \MET spectra, zoomed x-axis.]{\label{fig:ee_over_ff_di-EM_vs_dijet_pT_reweighting_zoom}\includegraphics[scale=0.3]{ee_over_ff_di-EM_vs_dijet_pT_reweighting_zoom}}
	\caption{\MET spectra of the reweighted $ee$ (81 GeV $\leq m_{\mathrm{ee}} <$ 101 GeV) and $\mathit{ff}$ control samples.  Blue squares indicate reweighting using the ECAL-only $p_{T}$ estimate; red triangles indicate reweighting using the PF $p_{T}$ estimate.  The full reweighting and normalization procedure is employed, along with $ee$ sideband subtraction (discussed at the end of this section).  Error bars include statistical, reweighting, and normalization error (see Sec.~\ref{sec:Systematic Errors}).}
	\label{fig:ee_vs_ff_di-EM_vs_dijet_pT_reweighting}
\end{figure}

The control sample event weights are defined as

\begin{eqnarray}
w_{ij} &=& \frac{N_{\mathrm{control}}}{N_{\gamma\gamma}}\frac{N_{\gamma\gamma}^{ij}}{N_{\mathrm{control}}^{ij}}
\end{eqnarray}
%
where $i$ runs over the number of di-EM $p_{T}$ bins, $j$ runs over the number of jet bins, $N_{\mathrm{control}}$ is the total number of events in the control sample, $N_{\gamma\gamma}$ is the total number of events in the $\gamma\gamma$ sample, $N_{\gamma\gamma}^{ij}$ is the number of $\gamma\gamma$ events in the $i^{\mathrm{th}}$ di-EM $p_{T}$ bin and $j^{\mathrm{th}}$ jet bin, and $N_{\mathrm{control}}^{ij}$ is the number of control sample events in the $i^{\mathrm{th}}$ di-EM $p_{T}$ bin and $j^{\mathrm{th}}$ jet bin.  The effect of the reweighting is more significant for the $ee$ sample than for the $\mathit{ff}$ sample, as shown in Figure~\ref{fig:reweighting_vs_no_reweighting}.

%include ee/gg and ff/gg agreement?
%include similar figures for MC?
\begin{figure}
	\centering
	\subfloat[$ee$ \MET spectra.]{\label{fig:ee_dijet_pT_and_Nj_vs_no_reweighting}\includegraphics[scale=0.3]{ee_dijet_pT_and_Nj_vs_no_reweighting}}
	\hspace{1cm}
	\subfloat[$\mathit{ff}$ \MET spectra.]{\label{fig:ff_dijet_pT_and_Nj_vs_no_reweighting}\includegraphics[scale=0.3]{ff_dijet_pT_and_Nj_vs_no_reweighting}}
	\\
	\subfloat[Ratio of $ee$ to $\mathit{ff}$ \MET spectra.]{\label{fig:ee_over_ff_dijet_pT_and_Nj_vs_no_reweighting}\includegraphics[scale=0.3]{ee_over_ff_dijet_pT_and_Nj_vs_no_reweighting}}
	\hspace{1cm}
	\subfloat[Ratio of $ee$ to $\mathit{ff}$ \MET spectra, zoomed x-axis.]{\label{fig:ee_over_ff_dijet_pT_and_Nj_vs_no_reweighting_zoom}\includegraphics[scale=0.3]{ee_over_ff_dijet_pT_and_Nj_vs_no_reweighting_zoom}}
	\caption{\MET spectra of the $ee$ (81 GeV $\leq m_{\mathrm{ee}} <$ 101 GeV) and $\mathit{ff}$ control samples.  Red triangles indicate full di-EM $p_{T}$ + number of jets reweighting; blue squares indicate no reweighting.  PF $p_{T}$ (cf. p.~\pageref{fig:ET_bias_vs_EMF}) is used to calculate the di-EM $p_{T}$.  The full normalization procedure is employed, along with $ee$ sideband subtraction (discussed at the end of this section).  Error bars include statistical, reweighting (where appropriate), and normalization error (see Sec.~\ref{sec:Systematic Errors}).}
	\label{fig:reweighting_vs_no_reweighting}
\end{figure}

The $ee$ sample contains a non-negligible background of $t\bar{t}$ events in which both $W$ bosons decay to electrons.  These events have significant real \MET from the two neutrinos (unlike the $\gamma\gamma$ events), and therefore inflate the background estimate at high \MET.  In order to remove the $t\bar{t}$ contribution from the $ee$ sample, a sideband subtraction method is employed.

Only events in the $ee$ sample with 81 GeV $\leq m_{\mathrm{ee}} <$ 101 GeV, where $m_{\mathrm{ee}}$ is the di-electron invariant mass, are used in the QCD background estimate.  This choice maximizes the ratio of $Z$ signal to background.  The sidebands used to estimate the background contribution within the $Z$ window are defined such that 71 GeV $\leq m_{\mathrm{ee}} <$ 81 GeV and 101 GeV $\leq m_{\mathrm{ee}} <$ 111 GeV.

The full reweighting procedure is applied to the $Z$ signal region and the two sideband regions independently.  Only $Z$ signal events are used in the calculation of the di-EM $p_{T}$ weights for the $Z$ signal region, and likewise only the events within a given sideband region are used in the calculation of the weights for that region.  Assuming a constant $t\bar{t}$ background shape, the resulting reweighted sideband \MET distributions are added together and subtracted from the reweighted $Z$ signal \MET distribution.  The sideband subtracted $Z$ signal \MET distribution is then normalized as discussed in Secs.~\ref{sec:Outline of the Procedure} and~\ref{sec:Normalization}.  The statistical and reweighting error from the sideband regions is propagated to the error on the final $ee$ QCD \MET estimate.

The di-EM $p_{T}$ weights for the two $ee$ sideband regions are shown in figure~\ref{fig:ee_sideband_dijet_pT_weights}.  The overall scale of the weights, as well as the trend with di-EM $p_{T}$, is similar for the two regions (except at high di-EM $p_{T}$, where the statistics are poor anyway).  Figure~\ref{fig:all_ee_MET_spectra} shows the \MET spectra for the two sideband regions and the $Z$ signal region after subtraction.  The shapes of the spectra indicate that the high-\MET $t\bar{t}$ tail, present in the sideband distributions, was successfully subtracted from the $Z$ signal distribution.

\begin{figure}
	\centering
	\subfloat[71 GeV $\leq m_{\mathrm{ee}} <$ 81 GeV, 0 jets.]{\label{fig:eeLowSideband_0_jet_dijet_pT_weights}\includegraphics[scale=0.2]{eeLowSideband_0_jet_dijet_pT_weights}}
	\hspace{1cm}
	\subfloat[71 GeV $\leq m_{\mathrm{ee}} <$ 81 GeV, 1 jet.]{\label{fig:eeLowSideband_1_jet_dijet_pT_weights}\includegraphics[scale=0.2]{eeLowSideband_1_jet_dijet_pT_weights}}
	\hspace{1cm}
	\subfloat[71 GeV $\leq m_{\mathrm{ee}} <$ 81 GeV, $\geq$ 2 jets.]{\label{fig:eeLowSideband_2_jet_dijet_pT_weights}\includegraphics[scale=0.2]{eeLowSideband_2_jet_dijet_pT_weights}}
	\\
	\subfloat[101 GeV $\leq m_{\mathrm{ee}} <$ 111 GeV, 0 jets.]{\label{fig:eeHighSideband_0_jet_dijet_pT_weights}\includegraphics[scale=0.2]{eeHighSideband_0_jet_dijet_pT_weights}}
	\hspace{1cm}
	\subfloat[101 GeV $\leq m_{\mathrm{ee}} <$ 111 GeV, 1 jet.]{\label{fig:eeHighSideband_1_jet_dijet_pT_weights}\includegraphics[scale=0.2]{eeHighSideband_1_jet_dijet_pT_weights}}
	\hspace{1cm}
	\subfloat[101 GeV $\leq m_{\mathrm{ee}} <$ 111 GeV, $\geq$ 2 jets.]{\label{fig:eeHighSideband_2_jet_dijet_pT_weights}\includegraphics[scale=0.2]{eeHighSideband_2_jet_dijet_pT_weights}}	
	\caption{$ee$ sideband di-EM $p_{T}$ weights for events with 0, 1, or $\geq 2$ jets (as in Table~\ref{tab:jet_definition_for_Nj_reweighting}).  Errors are statistical only.}
	\label{fig:ee_sideband_dijet_pT_spectra}
\end{figure}

\begin{figure}
	\centering
	\includegraphics[scale=0.3]{all_ee_MET_spectra}
	\caption{\MET spectra of the $ee$ sample for 71 GeV $\leq m_{\mathrm{ee}} <$ 81 GeV (red triangles), 81 GeV $\leq m_{\mathrm{ee}} <$ 101 GeV (black circles), and 101 GeV $\leq m_{\mathrm{ee}} <$ 111 GeV (blue squares).  The two sideband distributions (red and blue) are normalized to the area of the $Z$ signal distribution (black).  Errors on the sideband distributions are statistical only, while the error on the $Z$ signal distribution includes statistical, reweighting, and normalization error (see Sec.~\ref{sec:Systematic Errors}).}
	\label{fig:all_ee_MET_spectra}
\end{figure}

The $ee$ (81 GeV $\leq m_{\mathrm{ee}} <$ 101 GeV), $\mathit{ff}$, and $\gamma\gamma$ di-EM $p_{T}$ spectra for events with 0, 1, or $\geq 2$ jets (as in Table~\ref{tab:jet_definition_for_Nj_reweighting}) are shown in Figure~\ref{fig:dijet_pT}.  Broad humps in the $\mathit{ff}$ and $\gamma\gamma$ spectra are due to kinematic $\Delta R$ and $p_{T}$ turn-ons that are suppressed in the $ee$ sample due to the invariant mass cut.  Figure~\ref{fig:dijet_pT_weights} shows the weights applied to the $ee$ (81 GeV $\leq m_{\mathrm{ee}} <$ 101 GeV) and $\mathit{ff}$ \MET spectra as a function of di-EM $p_{T}$ and number of jets per event.

%zoom out?
\begin{figure}
	\centering
	\subfloat[0 jets.]{\label{fig:0_jet_dijet_pT}\includegraphics[scale=0.2]{0_jet_dijet_pT}}
	\hspace{1cm}
	\subfloat[1 jet.]{\label{fig:1_jet_dijet_pT}\includegraphics[scale=0.2]{1_jet_dijet_pT}}
	\hspace{1cm}
	\subfloat[$\geq$ 2 jets.]{\label{fig:2_jet_dijet_pT}\includegraphics[scale=0.2]{2_jet_dijet_pT}}
	\caption{$ee$ (81 GeV $\leq m_{\mathrm{ee}} <$ 101 GeV) (red triangles), $\mathit{ff}$ (blue squares), and $\gamma\gamma$ (black circles) di-EM $p_{T}$ spectra for events with 0, 1, or $\geq 2$ jets (as in Table~\ref{tab:jet_definition_for_Nj_reweighting}).  Errors are statistical only.  \textcolor{red}{\textbf{Zoom out the x-axis to show the full tail out to 500 GeV?}}}
	\label{fig:dijet_pT}
\end{figure}

%zoom in?
\begin{figure}
	\centering
	\subfloat[$ee$, 0 jets.]{\label{fig:ee_0_jet_dijet_pT_weights}\includegraphics[scale=0.2]{ee_0_jet_dijet_pT_weights}}
	\hspace{1cm}
	\subfloat[$ee$, 1 jet.]{\label{fig:ee_1_jet_dijet_pT_weights}\includegraphics[scale=0.2]{ee_1_jet_dijet_pT_weights}}
	\hspace{1cm}
	\subfloat[$ee$, $\geq$ 2 jets.]{\label{fig:ee_2_jet_dijet_pT_weights}\includegraphics[scale=0.2]{ee_2_jet_dijet_pT_weights}}
	\\
	\subfloat[$\mathit{ff}$, 0 jets.]{\label{fig:ff_0_jet_dijet_pT_weights}\includegraphics[scale=0.2]{ff_0_jet_dijet_pT_weights}}
	\hspace{1cm}
	\subfloat[$\mathit{ff}$, 1 jet.]{\label{fig:ff_1_jet_dijet_pT_weights}\includegraphics[scale=0.2]{ff_1_jet_dijet_pT_weights}}
	\hspace{1cm}
	\subfloat[$\mathit{ff}$, $\geq$ 2 jets.]{\label{fig:ff_2_jet_dijet_pT_weights}\includegraphics[scale=0.2]{ff_2_jet_dijet_pT_weights}}
	\caption{$ee$ (81 GeV $\leq m_{\mathrm{ee}} <$ 101 GeV) and $\mathit{ff}$ di-EM $p_{T}$ weights for events with 0, 1, or $\geq 2$ jets (as in Table~\ref{tab:jet_definition_for_Nj_reweighting}).  Errors are statistical only.  \textcolor{red}{\textbf{Zoom in the x-axis to hide large weights with large statistical errors?}}}
	\label{fig:dijet_pT_weights}
\end{figure}

\subsection{Normalization}
\label{sec:Normalization}

After reweighting, the \MET distributions of the QCD control samples are normalized to the \MET $< 20$ GeV region of the candidate $\gamma\gamma$ \MET spectrum, where signal contamination is low.  The normalization factor is ($N_{\gamma\gamma}^{\not\!\! E_{T} < 20 \mathrm{GeV}} - N_{e\gamma}^{\not\!\! E_{T} < 20 \mathrm{GeV}})/N_{\mathrm{control}}^{\not\!\! E_{T} < 20 \mathrm{GeV}}$, where $N_{e\gamma}^{\not\!\! E_{T} < 20 \mathrm{GeV}}$ is the expected number of electroweak background events with \MET $< 20$ GeV (discussed in Section~\ref{sec:Modeling the Electroweak Background}).

\section{Modeling the Electroweak Background}
\label{sec:Modeling the Electroweak Background}

$W\gamma$, $W$ + jet, and $t\bar{t}$ processes in which the $W$ decay electron is misidentified as a photon (due to a failure to properly associate a pixel seed to the electron candidate) can contribute significantly to the high-\MET tail of the $\gamma\gamma$ \MET spectrum.  To estimate this background, the $e\gamma$ sample, which is enriched in $W\rightarrow e\nu$ decays, is scaled by $f_{e\rightarrow\gamma}/(1 - f_{e\rightarrow\gamma})$, where $f_{e\rightarrow\gamma}$ is the rate at which electrons are misidentified as photons.  The derivation of this scaling factor comes from the two equations

\begin{eqnarray}
N_{e\gamma}^{W} &=& f_{e\rightarrow e}N_{W}\\
N_{\gamma\gamma}^{W} &=& (1 - f_{e\rightarrow e})N_{W}
\end{eqnarray}
%
where $N_{e\gamma}^{W}$ is the number of $W$ events in the $e\gamma$ sample in which the electron was correctly identified, $f_{e\rightarrow e}$ is the probability to correctly identify an electron, $N_{W}$ is the true number of triggered $W\rightarrow e\nu$ events, and $N_{\gamma\gamma}^{W}$ is the number of $W$ events in the $\gamma\gamma$ sample in which the electron was misidentified as a photon.  The contribution from $Z\rightarrow ee$ can be neglected (i.e. $f_{e\rightarrow\gamma}$ is small and the $Z$ contribution involves $f_{e\rightarrow\gamma}^{2}$, since both electrons have to be misidentified).  Since $f_{e\rightarrow e} = 1 - f_{e\rightarrow\gamma}$, solving for $N_{\gamma\gamma}^{W}$ gives

\begin{eqnarray}
N_{\gamma\gamma}^{W} = \frac{f_{e\rightarrow\gamma}}{1 - f_{e\rightarrow\gamma}}N_{e\gamma}^{W}
\end{eqnarray}

$f_{e\rightarrow\gamma}$ is measured by fitting the $Z$ peaks in the $ee$ and $e\gamma$ samples.  The number of $Z$ events fitted in the $ee$ and $e\gamma$ samples, respectively, is given by

\begin{eqnarray}
N_{ee}^{Z} &=& (1 - f_{e\rightarrow\gamma})^{2}N_{Z} \\
N_{e\gamma}^{Z} &=& 2f_{e\rightarrow\gamma}(1 - f_{e\rightarrow\gamma})N_{Z}
\end{eqnarray}
%
where $N_{Z}$ is the true number of triggered $Z\rightarrow ee$ events.  Solving for $f_{e\rightarrow\gamma}$ gives

\begin{eqnarray}
f_{e\rightarrow\gamma} = \frac{N_{e\gamma}^{Z}}{2N_{ee}^{Z} + N_{e\gamma}^{Z}}
\label{eq:feg}
\end{eqnarray}

A Crystal Ball function is used to model the $Z$ signal shape in both the $ee$ and $e\gamma$ samples, while an exponential convoluted with an error function (``RooCMSShape", see Sec.~\ref{sec:Tag_and_Probe_Method}) is used to model the background shape.  The fixed fit parameters are identical for the two samples, but the other parameters are allowed to float independently.  Table~\ref{tab:feg_fit_parameters} shows the values and ranges of the fixed and floating fit parameters, respectively.  \textcolor{red}{\textbf{Edit this to reflect the actual study once done.}}

\begin{table}[hcbp]
\caption{Parameter values for the signal and background PDFs for the $ee$ and $e\gamma$ samples.  When a bracketed range is given, the parameter is allowed to float within that range.  When a constant is given, the parameter is fixed to that constant.  \textcolor{red}{\textbf{Edit this to reflect the actual study once done.}}}
\centering
\begin{tabular}{|m{1.25cm}|m{1.25cm}|m{1.25cm}|m{1.25cm}|m{1.25cm}|m{1.25cm}|m{1.25cm}|m{1.25cm}|m{1.25cm}|}
\hline
& \multicolumn{4}{c|}{Crystal Ball fit parameters} & \multicolumn{4}{c|}{RooCMSShape fit parameters} \\
\hline
PDF & $\mu$ & $\sigma$ & $\alpha$ & n & $\mu$ & $\alpha$ & $\beta$ & $\gamma$ \\
\hline
$ee$ signal & [-1.0, 1.0] & [1.0, 3.0] & 0.87 & 97.0 & N/A & N/A & N/A & N/A \\
\hline
$e\gamma$ signal & [-1.0, 1.0] & [1.0, 3.0] & 0.73 & 134.9 & N/A & N/A & N/A & N/A \\
\hline
$ee$ background & N/A & N/A & N/A & N/A & 65.0 & 61.949 & 0.04750 & 0.01908 \\
\hline
$e\gamma$ background & N/A & N/A & N/A & N/A & $\alpha$ & [50.0, 100.0] & 0.065 & 0.048 \\
\hline
\end{tabular}
\label{tab:feg_fit_parameters}
\end{table}

Fits to the $ee$ and $e\gamma$ invariant mass spectra are shown in Figure~\ref{fig:feg_fits}.  \textcolor{red}{\textbf{Make these plots.}}  Figure~\ref{fig:feg_vs_pT_eta} indicates that the dependence of $f_{e\rightarrow\gamma}$ on the electron $p_{T}$ and $\eta$ is small.  Applying a $p_{T}$- and $\eta$-dependent misidentification rate (with $p_{T}$ and $\eta$ binned as in Fig.~\ref{fig:feg_vs_pT_eta}) makes only a \textcolor{red}{\textbf{XXX\%}} difference in the final electroweak background estimate with respect to a constant rate derived from all $ee$ and $e\gamma$ events, which is well within the statistical and systematic errors.  \textcolor{red}{\textbf{Prove this statement.}}  Therefore, the constant rate is used in the final electroweak background estimate.  \textcolor{red}{\textbf{Check this statement.}}

%\begin{figure}
%	\centering
%	\subfloat[$ee$.]{\label{fig:mee_fit}\includegraphics[scale=0.2]{mee_fit}}
%	\hspace{1cm}
%	\subfloat[$e\gamma$.]{\label{fig:meg_fit}\includegraphics[scale=0.2]{meg_fit}}
%	\caption{Fits to the $ee$ and $e\gamma$ invariant mass spectra using the Crystal Ball + RooCMSShape function described in the text and in Table~\ref{tab:feg_fit_parameters}.}
%	\label{fig:feg_fits}
%\end{figure}

%\begin{figure}
%	\centering
%	\subfloat[$f_{e\rightarrow\gamma}$ vs. electron $p_{T}$.]{\label{fig:feg_vs_pT}\includegraphics[scale=0.2]{feg_vs_pT}}
%	\hspace{1cm}
%	\subfloat[$f_{e\rightarrow\gamma}$ vs. electron $\eta$.]{\label{fig:feg_vs_eta}\includegraphics[scale=0.2]{feg_vs_eta}}
%	\caption{$f_{e\rightarrow\gamma}$ vs. electron $p_{T}$ and $\eta$.  Red squares indicate Drell-Yan and $W$ MC (see App.~\ref{chap:Monte Carlo Samples}); black circles indicate data.}
%	\label{fig:feg_vs_pT_eta}
%\end{figure}

The signal and background shape assumptions are the main sources of systematic error on $f_{e\rightarrow\gamma}$.  To assess the magnitude of this error, $f_{e\rightarrow\gamma}$ is recalculated using both linear and quadratic background shapes, and with a Crystal Ball + generated $Z$ signal shape (as used in Sec.~\ref{sec:Tag_and_Probe_Method}).  The largest difference from the nominal shape is taken as the error.  \textcolor{red}{\textbf{Check that this is how it was done, and do it yourself.  Also check the misidentification rate in MC with varied tracker radiation lengths to see if there is a dependence on the tracker density.}}

Using the integrals of the $Z$ fits shown in Fig.~\ref{feg_fits}, Eq.~\ref{eq:feg}, and the shape systematic discussed above, $f_{e\rightarrow\gamma}$ is calculated to be 0.015 $\pm$ 0.002(stat.) $\pm$ 0.005(syst.).  \textcolor{red}{\textbf{Replace with your calculated number.}}  The scaled $e\gamma$ MET distribution is shown in Figure~\ref{fig:eg_MET}.

\begin{figure}
	\centering
	\includegraphics[scale=0.4]{eg_MET}
	\caption{\MET spectrum of the $e\gamma$ sample after scaling by $f_{e\rightarrow\gamma}$.  The total error on $f_{e\rightarrow\gamma}$ is propagated to the total error on the electroweak background estimate.  \textcolor{red}{\textbf{How to properly treat the error when the same events are used in the $f_{e\rightarrow\gamma}$ calculation and in the $e\gamma$ sample?  Replace with figure using latest $f_{e\rightarrow\gamma}$, and include error bars.}}}
	\label{fig:eg_MET}
\end{figure}

In the 36 $\mbox{pb}^{-1}$ version of this analysis \cite{CMS_GMSB_35pb-1}, it was shown that the $ee$ sample could accurately predict the QCD and real $Z$ contribution to the $e\gamma$ sample at low \MET, and that the expectation from $W\rightarrow e\nu$ MC accounted for the remaining $W$ contribution at high \MET.  A plot of the \MET distributions of the 2010 $e\gamma$ sample and the predicted components is shown in Figure~\ref{fig:35pb-1_eg_closure_test}.  \textcolor{red}{\textbf{Repeat for current selection?}}  This exercise helps to validate both the QCD and electroweak background prediction methods.

\begin{figure}
	\centering
	\includegraphics[scale=0.4]{35pb-1_eg_closure_test}
	\caption{\MET spectrum of the $e\gamma$ sample in 35 $\mbox{pb}^{-1}$ of 2010 LHC data scaled by the 2010 measured $f_{e\rightarrow\gamma}$ (black dots), QCD and real $Z$ predicted background from the 2010 $ee$ sample (solid orange line), MC $W$ + jet estimate (dash-dotted purple line), and MC $W\gamma$ estimate (dashed blue line).  The total $e\gamma$ prediction (red band) is the sum of the $ee$, $W$ + jet, and $W\gamma$ predictions.  Reprinted from Fig. 2 of ref. \cite{CMS_GMSB_35pb-1}.}
	\label{fig:35pb-1_eg_closure_test}
\end{figure}

\section{Systematic Errors}
\label{sec:Systematic Errors}

%statistical errors
%	- normalization
%	- reweighting (incl. sidebands for ee)

%systematic errors
%	- EW bkg. estimate from feg
%	- JES (folded into reweighting, see below for details)
%	- bin definitions (possibly covered by 1000 toys procedure)
%	- ee background shape assumption
%		- get true shape from MC
%		- get corroborating estimate from OF subtraction
%	- slightly different hadronic activity between ff, ee, and gg

%systematic studies
%variations in shape of dijet pT spectra and jet bin migration due to statistics and JES
%show that effect of rebinning the dijet pT weights is covered by the 1000-toys procedure
%derive a systematic from subtracting the ee sidebands with different relative weights
%	- also how do we know we're not undersubtracting at high MET -- show table of subtracted stuff per MET bin
%corrected vs. uncorrected MET
%systematic due to different hadronic activity (explain how it is in data and MC)
%HT, MHT, Nj ee vs. ff before and after reweighting

%ee as cross-check
%slightly better MET resolution than gg or ff
%MC closure
%...

\subsection{Jet Energy Scale Uncertainty}

The dijet \pT reweighting method utilizes jets corrected for imperfect calorimeter response (see Sec.~\ref{sec:Jets and Missing Transverse Energy} for a description of the jet reconstruction and correction procedure).  Since the applied jet energy scale (JES) factor has an error associated to it due to the limitations of the JES derivation (\cite{CMS_JES_paper} and Sec.~\ref{sec:Jets and Missing Transverse Energy}), this uncertainty must be propagated to the uncertainty on the dijet \pT weights.

The JES contribution to the dijet \pT weights is estimated by performing 1000 pseudo-experiments on each of the $\gamma\gamma$ and ff samples.  For the purpose of estimating the JES error, the results of the true experiment may be thought of as a set of measurements:

\begin{itemize}
  \item The set of \textbf{uncorrected jet 4-vectors} corresponding to the \textbf{leading EM object} in the $\gamma\gamma$ sample $\left\{\mbox{p}_{\mbox{j}1}^{\mu1}, \mbox{p}_{\mbox{j}1}^{\mu2},...,\mbox{p}_{\mbox{j}1}^{\mu \mbox{N}_{\gamma\gamma}}\right\}$
  \item The set of \textbf{uncorrected jet 4-vectors} corresponding to the \textbf{trailing EM object} in the $\gamma\gamma$ sample $\left\{\mbox{p}_{\mbox{j}2}^{\mu1}, \mbox{p}_{\mbox{j}2}^{\mu2},...,\mbox{p}_{\mbox{j}2}^{\mu \mbox{N}_{\gamma\gamma}}\right\}$
  \item The set of \textbf{JES} accompanying the uncorrected jet 4-vectors corresponding to the \textbf{leading EM object} in the $\gamma\gamma$ sample $\left\{\mbox{c}_{\mbox{j}1}^{1}, \mbox{c}_{\mbox{j}1}^{2},...,\mbox{c}_{\mbox{j}1}^{\mbox{N}_{\gamma\gamma}}\right\}$
  \item The set of \textbf{JES} accompanying the uncorrected jet 4-vectors corresponding to the \textbf{trailing EM object} in the $\gamma\gamma$ sample $\left\{\mbox{c}_{\mbox{j}2}^{1}, \mbox{c}_{\mbox{j}2}^{2},...,\mbox{c}_{\mbox{j}2}^{\mbox{N}_{\gamma\gamma}}\right\}$
  \item The set of \textbf{JES uncertainties} accompanying the uncorrected jet 4-vectors corresponding to the \textbf{leading EM object} in the $\gamma\gamma$ sample $\left\{\sigma_{\mbox{cj}1}^{1}, \sigma_{\mbox{cj}1}^{2},...,\sigma_{\mbox{cj}1}^{\mbox{N}_{\gamma\gamma}}\right\}$
  \item The set of \textbf{JES uncertainties} accompanying the uncorrected jet 4-vectors corresponding to the \textbf{trailing EM object} in the $\gamma\gamma$ sample $\left\{\sigma_{\mbox{cj}2}^{1}, \sigma_{\mbox{cj}2}^{2},...,\sigma_{\mbox{cj}2}^{\mbox{N}_{\gamma\gamma}}\right\}$
  \item The set of \textbf{uncorrected jet 4-vectors} corresponding to the \textbf{leading EM object} in the ff sample $\left\{\mbox{p}_{\mbox{j}1}^{\mu1}, \mbox{p}_{\mbox{j}1}^{\mu2},...,\mbox{p}_{\mbox{j}1}^{\mu \mbox{N}_{\mbox{ff}}}\right\}$
  \item The set of \textbf{uncorrected jet 4-vectors} corresponding to the \textbf{trailing EM object} in the ff sample $\left\{\mbox{p}_{\mbox{j}2}^{\mu1}, \mbox{p}_{\mbox{j}2}^{\mu2},...,\mbox{p}_{\mbox{j}2}^{\mu \mbox{N}_{\mbox{ff}}}\right\}$
  \item The set of \textbf{JES} accompanying the uncorrected jet 4-vectors corresponding to the \textbf{leading EM object} in the ff sample $\left\{\mbox{c}_{\mbox{j}1}^{1}, \mbox{c}_{\mbox{j}1}^{2},...,\mbox{c}_{\mbox{j}1}^{\mbox{N}_{\mbox{ff}}}\right\}$
  \item The set of \textbf{JES} accompanying the uncorrected jet 4-vectors corresponding to the \textbf{trailing EM object} in the ff sample $\left\{\mbox{c}_{\mbox{j}2}^{1}, \mbox{c}_{\mbox{j}2}^{2},...,\mbox{c}_{\mbox{j}2}^{\mbox{N}_{\mbox{ff}}}\right\}$
  \item The set of \textbf{JES uncertainties} accompanying the uncorrected jet 4-vectors corresponding to the \textbf{leading EM object} in the ff sample $\left\{\sigma_{\mbox{cj}1}^{1}, \sigma_{\mbox{cj}1}^{2},...,\sigma_{\mbox{cj}1}^{\mbox{N}_{\mbox{ff}}}\right\}$
  \item The set of \textbf{JES uncertainties} accompanying the uncorrected jet 4-vectors corresponding to the \textbf{trailing EM object} in the ff sample $\left\{\sigma_{\mbox{cj}2}^{1}, \sigma_{\mbox{cj}2}^{2},...,\sigma_{\mbox{cj}2}^{\mbox{N}_{\mbox{ff}}}\right\}$
\end{itemize}
%
From these measurements, the $\gamma\gamma$ and ff dijet \pT spectra and the resulting ff dijet weights can be calculated.  In each of the 1000 pseudo-experiments, a new set of JES factors is generated according to the measured JES uncertainties, and new dijet \pT spectra and weights are subsequently calculated.  The spread of the 1000 weights (binned in dijet \pT) is taken as the error due to JES uncertainty.  The total error on the weights is the quadrature sum of the JES error and the statistical error, and is propagated to the error on the final \MET measurement via a similar pseudo-experiment procedure described in Sec.~\ref{sec:Statistical Uncertainty in the ff or ee Weights}.\footnote{The \MET is uncorrected and therefore its central value per event is unaffected by a change in the JES.}

%do this check
If the JES uncertainty were to cause the jet energy to be reconstructed below the 20 GeV ntuple cut, there could be a small error or bias in the \MET introduced due to EM-matched jets falling below the matching threshold.  The percentage of jets lost due to jet \ET matching threshold has been checked in data, and found to be X\% (X\% of events).  Furthermore, the trailing EM \ET cut is 25 GeV/c, implying that the JES would have to be mis-measured by at least 20\% to fall below the jet matching threshold.  Since the typical JES uncertainty is no more than 5\%, a mis-measurement of this type is a 4$\sigma$ event and should occur in only 0.1\% of cases.  As expected, this effect is negligible, as shown in Figure X.

%\begin{figure}
%	\centering
%	\subfloat[ff dijet \pT weights including effect of jets falling below the matching threshold.]{\label{fig:dummy}\includegraphics[scale=0.7]{dummy}}
%	\hspace{1cm}
%	\subfloat[ff dijet \pT weights neglecting effect of jets falling below the matching threshold.]{\label{fig:dummy}\includegraphics[scale=0.7]{dummy}}
%	\hspace{1cm}
%	\subfloat[Percentage difference between JES errors shown in (a) and (b).]{\label{fig:dummy}\includegraphics[scale=0.7]{dummy}}
%	\caption{ff dijet \pT weights with JES error.}
%\end{figure}

\section{Results}
\label{sec:Results}

%results
%final MET plot
%final number in signal region with all stat. and syst. errors included
%reminder table of all uncertainties that are described more thoroughly in previous sections
%MC closure test
%expectation from GGM models (1 that can be excluded, 1 that can't)

\end{document}