\documentclass[dissertation.tex]{subfiles} 
\begin{document}

\begin{abstract}

\thispagestyle{myheadings}
\markright{\hfill}

This thesis presents a search for evidence of new particle production in the two-photon + jets + missing transverse energy final state using the 2011 Large Hadron Collider proton-proton data, at a center-of-mass energy of 7 TeV, collected by the Compact Muon Solenoid experiment.  The distinctive signature of two photons, jets, and a momentum imbalance in the plane transverse to the proton beam direction is chiefly motivated by the theory of supersymmetry, which may provide a solution to the hierarchy problem in particle physics.  The instrumental background from ordinary quantum chromodynamics and $W\rightarrow e\nu$ production dominates the total background estimate, and is measured from the data.  This background estimate, as well as the high efficiency ($\sim80\%$) to select and identify two high-energy photons in the data, relies on the superior energy resolution of the lead tungstate crystal electromagnetic calorimeter of the Compact Muon Solenoid as compared to the hadronic energy resolution.  Signal efficiencies for gauge-mediated supersymmetric models, ranging from a few percent to approximately 25\%, are taken from Monte Carlo simulation.  No evidence of new particles is found in a dataset of 4.7 $\mbox{fb}^{-1}$ of integrated luminosity.  Upper limits between 5 and 15 fb on the cross sections of various new physics models are set at the 95\% confidence level.  The upper limits measured in this search are the most stringent to date for gauge-mediated models of supersymmetry.

\end{abstract}

\end{document}